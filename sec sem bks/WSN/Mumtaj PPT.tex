\documentclass[11pt]{class}
\usetheme{Warsaw}

\usepackage[utf8]{inputenc}
\usepackage[T1]{fontenc}
\usepackage{amsfonts}

\author [Mumtaj]{\scriptsize by \\[1cm] color Ian F.Akyildiz \\Mehmet Can Vuran}
\title{WIRELESS SENSOR AND ACTOR NETWORKS}
\begin{document}
	\maketitle
	
	\begin{frame}
		\frametitle{WSAN}
		\begin{enumerate}
			\item   WSANs consists of two classes of components: sensors and actors.
			\item  The phenomena of sensing and acting are performed by sensor and actor nodes
			\item  These two classes of components have the following distinct characteristics:.
			\begin{itemize}
			\item Sensor :
				\begin{itemize}
					\item low-cost, low-power devices with limited sensing
				    \item A sensor node may consist of multiple sensors and observe
				    different physical phenomena through these sensors
			\item Actor :
				\begin{itemize}
					\item Resource-rich nodes equipped with higher processing capabilities, higher transmission power, and potentially longer battery life
					\item Actors may be mobile improves the effective areas in which they can act
					\item An actor node may be embedded with different actuators to perform different tasks
					\item By communicating multiple actors can coordinate to decide on appropriate actions based on information received from multiple sensors.
				\end{itemize}
				
				
			\end{enumerate}
			
		\end{frame}
		\begin{frame}
			\frametitle{WSAN's CHARACTERISTICS}
			\begin{enumerate}
				\item WSANs have the following unique characteristics:
					\begin{itemize}
						\item Heterogeneity: 
						\begin{itemize}
							\item WSANs consist of heterogeneous components including low-end sensor nodes and more powerful actor nodes.
							\item heterogeneous sets of network protocols are required
						\end{itemize}
						\item Real Time requirement :
						\begin{itemize}
							\item  WSANs are essentially closed-loop systems, where decisions are made according to the input from the sensors
							\item Example in the fire monitoring application, if a fire is detected, the sensors should relay this information to the appropriate sprinklers to immediately put out the fire. 
							\item The collected and delivered sensor data must still be valid at the time of acting
						\end{itemize}
						\item Co ordination
							\begin{itemize}
								\item In WSANs, new networking phenomena called sensor–actor and actor–actor coordination 
								\item 
							\end{itemize}
						
							\end{enumerate}
						\end{frame}
						\begin{frame}
							\frametitle{Characteristics of WSANs}
							\begin{enumerate}
								\item WSANs consist of sensor and actor nodes which collect data from the environment
								\item perform appropriate actions based on these collected data
							\end{enumerate}
								\end{frame}
								\begin{frame}
								\frametitle{NETWORK ARCHITECTURE}
								\begin{enumerate}
									\item WSANs can be organized into two main architectures. 
								\begin{itemize}
									\item Automated architecture:
									\begin{itemize}
										\item  sensors send their observations to actors.
										\item  The actors coordinate among each other to decide on the appropriate action and perform task assignment.
										\item  Due to the non-existence of a central controller, e.g., sink or human interaction, this architecture is called automated.
										\item  The observations are distributed among actors and they need to coordinate to make decisions
							            \item Advantages: 
							            \begin{itemize}
							            	\item Lower latency : communication of the observed information from sensors to actors has a much smaller latency compared to sending this information to the sink.
							            	\item Longer Network Lifetime: The automated architecture results in localized communication, where the sensor observations are delivered to the nearby actors
							            	This load will be generally small
							            	As a result, the automated architecture will have a longer lifetime than the semi-automated
							            	architecture.
							            	\item Lower network resource consumption: the number of sensors involved in the communication process in the automated architecture is smaller than that in the semi-automated architecture.
							            	 The network load is distributed and is based on the locations of the events.
							            	  This results in network resource (i.e., energy, bandwidth, etc.) savings inWSANs.
							            \end{itemize}
								\end{itemize}
								\item Semi-automated architecture: 
								\begin{itemize}
									\item In this the sink, i.e., central controller collects data and
									coordinates the acting process.
									\item Disadvantage: the semi-automated architecture is prone to single points of failure, where the failure of the sink affects all the tasks in the network
								\end{itemize} 
							\end{itemize}
						\end{enumerate}
							\end{frame}
							\begin{frame}
								\frametitle{PHYSICAL ARCHITECTURE}
								\begin{enumerate}
									\item 
								\end{enumerate}
								\end{frame}
									\begin{frame}
										\frametitle{Sensor–Actor Coordination}
										\begin{enumerate}
											\item The most important requirement of sensor–actor coordination is to provide low communication delay due to the proximity of sensors and actors.
											\item The three main
											challenges arise for this coordination:
											\begin{itemize}
												\item Requirements of sensor–actor communication
												\item Actor selection
												\item Communication technique
											\end{itemize}
										\end{enumerate}
									\end{frame}
									\begin{frame}
										\frametitle{Requirements of Sensor–Actor Communication}
										\begin{enumerate}
											\item The main requirements of for sensor–actor communication for WSANs are 
											\begin{itemize}
												\item Real-time bounds
												\item Energy Efficiency
												\item Event ordering
												\item Event synchronization
												\item Actor selection support
											\end{itemize}
										\end{enumerate}
									\end{frame}
									\begin{frame}
										\frametitle{ACTOR SELECTION}
										\begin{enumerate}
											\item there are four alternatives for actor selection
											\begin{itemize}
												\item Minimum actor set: A minimal set of actors to cover the event region.												• \item Minimum sensor set: The minimum number of sensors to report the sensed event.
												\item Minimum actor and sensor set: Both cases above.
												\item Area-based set: The entire set of actors and sensors in the vicinity of the region.
										\end{itemize}
										\end{enumerate}
									\end{frame}
									\begin{frame}
										\frametitle{Optimal Solution}
										\begin{enumerate}
											\item 
											\begin{itemize}
												\item Latency bound
												\item Expired and reliable packets
												\item Event reliability
												\item Event reliability threshold
												\item Lack of reliability
											\end{itemize}
										\end{enumerate}
									\end{frame}
									
										\begin{frame}
											\frametitle{Distributed Event-Driven Clustering and Routing (DECR) Protocol}
											\begin{enumerate}
												\item 
											\end{enumerate}
										\end{frame}
										
										\begin{frame}
											\frametitle{PERFORMANCE}
											\begin{enumerate}
												\item 
											\end{enumerate}
										\end{frame}
											\begin{frame}
												\frametitle{Challenges for Sensor–Actor Coordination}
												\begin{enumerate}
													\item 
												\end{enumerate}
											\end{frame}
										
										\begin{frame}
											\frametitle{Actor–Actor Coordination}
											\begin{enumerate}
												\item Actor–actor communication occurs in the following situations
											\begin{itemize}
												\item Task distribution
												\item Event information exchange
												\item Task co ordination
												\item Synchronization
												\item Task execution
												\item Event information delay
											\end{itemize}
											\item The task assignment problems in WSANs can be classified as
											\begin{itemize}
												\item Single actor task (SAT) vs. multi-actor task (MAT):
												\begin{itemize}
													\item 
												\end{itemize}
												\item Centralized decision (CD) vs. distributed decision (DD):
												\begin{itemize}
													\item 
												\end{itemize}
											\end{itemize}
											\end{enumerate}
										\end{frame}
										
										\begin{frame}
											\frametitle{Task Assignment}
											\begin{enumerate}
												\item MAT
												\item SAT
											\end{enumerate}
										\end{frame}
										\begin{frame}
											\frametitle{Optimal Solution}
											\begin{enumerate}
												\item 
												\begin{itemize}
													\item Action area
													\item Action range
													\item Action completion time
													\item Action completion bound 
												\end{itemize}
											
											\end{enumerate}
										\end{frame}
										
										\begin{frame}
											\frametitle{Localized Auction Protocol}
											\begin{enumerate}
												\item 
												\begin{itemize}
													\item Seller
													\item Auctioneer
													\item Buyer
												\end{itemize}
												
											\end{enumerate}
										\end{frame}
										\begin{frame}
											\frametitle{PERFORMANCE EVALUATION}
											\begin{enumerate}
												\item 
												\item 
											\end{enumerate}
										\end{frame}
										\begin{frame}
											\frametitle{PERFORMANCE EVALUATION}
											\begin{enumerate}
												\item 
												\item 
											\end{enumerate}
										\end{frame}
										\begin{frame}
											\frametitle{PERFORMANCE EVALUATION}
											\begin{enumerate}
												\item 
												\item 
											\end{enumerate}
										\end{frame}
										\begin{frame}
											\frametitle{PERFORMANCE EVALUATION}
											\begin{enumerate}
												\item 
												\item 
											\end{enumerate}
										\end{frame}
										\begin{frame}
											\frametitle{PERFORMANCE EVALUATION}
											\begin{enumerate}
												\item 
												\item 
											\end{enumerate}
										\end{frame}
										
										\begin{frame}
											\frametitle{Challenges for Actor–Actor Coordination}
											\begin{enumerate}
												\item 
												\item 
											\end{enumerate}
										\end{frame}
										
										\begin{frame}
											\frametitle{WSAN PROTOCAL STACK}
											\begin{enumerate}
												\item 
												\begin{itemize}
													\item Communiaction plane 
													\item Coordination plane
													\item Management plane
												\end{itemize}
												 
											\end{enumerate}
										\end{frame}
											\begin{frame}
												\frametitle{MANAGEMENT PLANE}
												\begin{enumerate}
													\item 
													\begin{itemize}
														\item Power management plane 
														\item Mobility management plane
														\item Fault Management plane
													\end{itemize}
													
												\end{enumerate}
											\end{frame}
											\begin{frame}
												\frametitle{CO ORDINATION PLANE}
												\begin{enumerate}
													\item 
													\begin{itemize}
														\item  
														\item 
														\item 
													\end{itemize}
													
												\end{enumerate}
											\end{frame}
											\begin{frame}
												\frametitle{MANAGEMENT PLANE}
												\begin{enumerate}
													\item 
													\begin{itemize}
														\item Communiaction plane 
														\item Coordination plane
														\item Management plane
													\end{itemize}
													
												\end{enumerate}
											\end{frame}
												\begin{frame}
													\frametitle{COMMUNICATION PLANE}
													\begin{enumerate}
														\item 
														\begin{itemize}
															\item  
															\item 
															\item 
														\end{itemize}
														
													\end{enumerate}
												\end{frame}